%# -*- coding: utf-8-unix -*-

\chapter{量子物理}
\label{chap3}
\section{康普顿散射}
\begin{myprop}{康普顿散射公式}{1}
	 康普顿散射是指当X射线或伽马射线的光子跟物质相互作用,因失去能量而导致波长变长的现象,其波长变化量公式如下
	\[
		\lambda_f-\lambda_i=\frac{h}{m_e c}(1-\cos \theta)	
	\]
\end{myprop}
\begin{proof}
	其中$\underrightarrow{P}$表示四维动量, $i/f$代表initial/final, $\theta$ 是光子偏转角$\left( \text{能、动量守恒也能做} \right)$ 
	\[
		\begin{gathered}
		\underrightarrow{P_{\gamma i}}+\underrightarrow{P_{ei}}=\underrightarrow{P_{\gamma f}}+\underrightarrow{P_{ef}}\,\,   \Rightarrow \,\,   \underrightarrow{P_{\gamma i}}+\underrightarrow{P_{ei}}-\underrightarrow{P_{\gamma f}}=\underrightarrow{P_{ef}}\,\,   \Rightarrow \,\,   (\underrightarrow{P_{\gamma i}}+\underrightarrow{P_{ei}}-\underrightarrow{P_{\gamma f}})^2=(\underrightarrow{P_{ef}})^2
		\\
		\Rightarrow \,\,   0+\left( -m_ec^2 \right) +0+2\left( -\underrightarrow{P_{\gamma i}}\cdot \underrightarrow{P_{\gamma f}}+\underrightarrow{P_{ei}}\cdot \left( \underrightarrow{P_{\gamma i}}-\underrightarrow{P_{\gamma f}} \right) \right) =-m_ec^2
		\\
		\Rightarrow \,\,   \underrightarrow{P_{\gamma i}}\cdot \underrightarrow{P_{\gamma f}}=\underrightarrow{P_{ei}}\cdot \left( \underrightarrow{P_{\gamma i}}-\underrightarrow{P_{\gamma f}} \right) \,\,  \Rightarrow \,\,   \frac{h^2}{\lambda _i\lambda _f}\left( \cos \theta -1 \right) =-m_ec\left( \frac{h}{\lambda _i}-\frac{h}{\lambda _f} \right) 	
		\end{gathered}
	\]	
\end{proof}

\begin{remark}
	应用康普顿散射公式和能量守恒,可以快速得到电子的动能和散射角度
	\[
		\begin{cases}
			E_{k}=\dfrac{hc}{\lambda_0}-\dfrac{hc}{\lambda_0+\Delta\lambda}=\dfrac{\dfrac{h^2c^2}{m_e}(1-\cos \theta)}{\lambda_0(\lambda_0+\dfrac{hc}{m_e}(1-\cos \theta))}\\
			\tan \varphi = \dfrac{\lambda_0}{\dfrac{hc}{m_e}+\lambda_0}\dfrac{\sin \theta}{1-\cos \theta}=\dfrac{\lambda_0}{\dfrac{hc}{m_e}+\lambda_0}\cot \left(\dfrac{\theta}{2}\right)
		\end{cases}
	\]
\end{remark}

\section{普朗克黑体辐射}
\[
	\left. \begin{array}{c}
		\text{瑞丽}-\text{金斯公式(长波)}: \dfrac{8\pi \nu ^2}{c^3}k_BT\\
		\text{维恩公式(短波)}: \dfrac{8\pi \nu ^2}{c^3}\cdot \dfrac{h\nu}{e^{h\nu /k_BT}}\\
	\end{array} \right\} \xRightarrow[\text{爱因斯坦受激辐射理论}]{\text{普朗克凑出}}\text{普朗克公式}:\frac{8\pi \nu ^2}{c^3}\cdot \frac{h\nu}{e^{h\nu /k_BT}-1}\,\,	
\]


\section{波函数、算符和对易性}

\begin{mydef}{波函数统计解释和归一性}{1}
	\[
		|\varPsi (\vec{r},t)|^2=\varPsi ^*(\vec{r},t)\times \varPsi (\vec{r},t)\text{代表}t\text{时刻}, \vec{r}\text{附近的概率密度。}	
	\]
	\[
		\text{概率的归一性}: 1=\int_{\text{全空间}}{dP}=\int_{\text{全空间}}{|\varPsi (\vec{r},t)|^2dP}
	\]
\end{mydef}

\begin{mydef}{算符的引入}{1}
	\[
		\text{对自由粒子}
		\begin{cases}
		 \varPsi (\vec{r},t)=\varPsi _0e^{i\left( \vec{k}\cdot \vec{r}-\omega t \right)}\xlongequal{\text{代入}\vec{p}=\hbar \vec{k},E=\hbar \omega}\varPsi _0e^{-\frac{i}{\hbar}(Et-\vec{p}\cdot \vec{r})}\\
		-i\hbar \dfrac{\partial}{\partial x}\,\,\varPsi (\vec{r},t)=-i\hbar \varPsi _0e^{-\frac{i}{\hbar}(Et-\vec{p}\cdot \vec{r})}\frac{\partial}{\partial x}\left( -\dfrac{i}{\hbar}(Et-\vec{p}\cdot \vec{r}) \right) =p_x\varPsi (\vec{r},t)
		\end{cases}
	\]
	\[
		\text{因此,我们把}-i\hbar \frac{\partial}{\partial x}\text{称作}x\text{方向的动量算符,记作}\hat{p}_x=-i\hbar \frac{\partial}{\partial x}	
	\]
	\[
		\text{同样的,我们有}\boldsymbol{\hat{p}}=\left[ \begin{array}{c}
			\hat{p}_x\\
			\hat{p}_y\\
			\hat{p}_z\\
		\end{array} \right] =\left[ \begin{array}{c}
			-i\hbar \dfrac{\partial}{\partial x}\\
			-i\hbar \dfrac{\partial}{\partial y}\\
			-i\hbar \dfrac{\partial}{\partial z}\\
		\end{array} \right] =-i\hbar \nabla    \text{称作动量算符}	
	\]
\end{mydef}
\par 需要指出的是,虽然我们选择从自由粒子导出,但这些算符对于不同的波函数是普适的。对于其它的算符,下面给出一些例子
\begin{example}
尝试给出非相对论动能算符、坐标表象下的系统总能量算符、轨道角动量算符、轨道角动量平方算符。
\soln

	% \[
	% 	\\
	% \]
	% \[
	% 	\\
	% \]
	% \[
	% 	\\
	% \]
	% \[
	% 	\\
	% \]
	\[
		\text{非相对论动能算符}\hat{T}=-\frac{\boldsymbol{\hat{p}}^2}{2m}=-\frac{\hbar ^2}{2m}\vec{\nabla}^2\,\,   
	\]
	\[
		\text{坐标表象下的系统总能量算符}\hat{H}=\hat{T}+\hat{V}(\boldsymbol{r})
	\]
	\[
	\text{角动量算符}\boldsymbol{\hat{l}}=\boldsymbol{\hat{r}}\times \boldsymbol{\hat{p}}=\left[ \begin{array}{c}
	-i\hbar \left( y\dfrac{\partial}{\partial z}-z\dfrac{\partial}{\partial y} \right)\\
	-i\hbar \left( z\dfrac{\partial}{\partial x}-x\dfrac{\partial}{\partial z} \right)\\
	-i\hbar \left( x\dfrac{\partial}{\partial y}-y\dfrac{\partial}{\partial x} \right)\\
	\end{array} \right] =\left[ \begin{array}{c}
	i\hbar \left( \sin \varphi \dfrac{\partial}{\partial \theta}+\cot \theta \cos \varphi \dfrac{\partial}{\partial \varphi} \right)\\
	i\hbar \left( -\cos \varphi \dfrac{\partial}{\partial \theta}+\cot \theta \sin \varphi \dfrac{\partial}{\partial \varphi} \right)\\
	-i\hbar \dfrac{\partial}{\partial \varphi}
	\end{array} \right] 
	\]
	\[
	\text{角动量平方算符} \boldsymbol{\hat{l}}^2\equiv \boldsymbol{\hat{l}}_{x}^{2}+\boldsymbol{\hat{l}}_{y}^{2}+\boldsymbol{\hat{l}}_{z}^{2}=\left( \boldsymbol{\hat{p}}^2-\boldsymbol{\hat{p}}_{r}^{2} \right) \boldsymbol{\hat{r}}^2=-\hbar ^2\left[ \frac{1}{\sin \theta}\frac{\partial}{\partial \theta}\left( \sin \theta \frac{\partial}{\partial \theta} \right) +\frac{1}{\sin ^2\theta}\frac{\partial ^2}{\partial \varphi ^2} \right] 
	\]
\end{example}
\begin{mydef}{算符性质}{1}
	\begin{itemize}
		\item 标量 $A$ 与 $B: A B=B A$
		\item 矢量 $\vec{A}$ 与 $\vec{B}: \vec{A} \cdot \vec{B}=\vec{B} \cdot \vec{A} \quad \vec{A} \times \vec{B}=-\vec{B} \times \vec{A}$
		\item 算符 $\hat{p}$ 与 $\hat{x}$的关系
		\[
			\hat{p}_x[x \psi(x)]=-i \hbar \frac{\partial}{\partial x}[x \psi(x)]=-i \hbar \psi(x)-i \hbar x \frac{\partial}{\partial x} \psi(x)=-i \hbar \psi(x)+x \hat{p}_x \psi(x)
		\]
		由于波函数 $\psi(x)$ 任意,可以略去不写,记作 $\hat{p}_x \hat{x}=-i \hbar+\hat{x} \hat{x}_x \Leftrightarrow \hat{p}_x \hat{x}-\hat{x} \hat{p}_x=-i \hbar$ 我们将用 $\left[\hat{p}_x,\hat{x}\right]$ 来表示 $\hat{p}_x \hat{x}-\hat{x} \hat{p}_x$,则对易关系表示为: $\left[\hat{p}_x, \hat{x}\right]=-i \hbar$。
	\end{itemize}
\end{mydef}

\begin{mythm}{不确定关系}{1}
	理论上可以证明,若用 $\sigma_A$ 表达物理量 $A$ 的标准差,$\hat{A}$ 表示 $A$ 对应的算符,则有
	\[
		\sigma_A \sigma_B \geqslant\left|\frac{\langle[\hat{A}, \hat{B}]\rangle}{2 i}\right|
	\]
\end{mythm}
\begin{example}
	取 $A, B$ 为 $p_x, x$,代入定理3.6推导测不准原理。
	\soln 

	\par 则 $\left[\hat{p}_x,\hat{x}\right]=-i \hbar$ 是我们已知的结果, 代入上式得到 $\sigma_{p_x} \sigma_x \geqslant\left|\dfrac{-i \hbar}{2 i}\right|=\dfrac{\hbar}{2}$,这就是我们熟悉的 $p_x, x$ 的不确定性关系。
	% \[
	% 	\\
	% \]
\end{example}
