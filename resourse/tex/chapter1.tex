%# -*- coding: utf-8-unix -*-
%%==================================================
\chapter{几何光学}





\section{临界光线}

\begin{myprop}{ 最小偏向角}{}
	当一束光在顶角为$\alpha$、折射率为$n$的三棱镜中发生两次折射时,其偏向角有极小值,当入射角和出射角相等时取极小值,最小偏向角满足的方程如下
    \[
        n\sin\left(\dfrac{A}{2}\right)=\sin\left(\dfrac{\delta_{\min}+A}{2}\right)
    \]
\end{myprop}
该性质的一个直观想法是,由于\textbf{光路可逆原理},对称角附近偏离相同角度时互为可逆光,总偏向角不变,故为极值,至于是极大值还是极小值需要计算二阶导数。
\begin{proof}
    设第一次折射对应的入射角和出射角分别为$i_1,r_1$,第二次折射对应的入射角和出射角分别为$i_2,r_2$,有几何关系$\dfrac{\pi}{2}-r_1+\dfrac
    {\pi}{2}-i_2+A=\pi$,总偏折角度
    \[
        \Delta = i_1-r_1+r_2-i_2=i_1+r_2-A
    \]
    由折射定律关系$\sin i_1=n\sin r_1,n\sin i_2=\sin r_2$,将总偏折角度代换为$r_1$的函数
    \[
        \Delta(r_1)=\arcsin(n\sin r_1)+\arcsin(n\sin (A-r_1))-A    
    \]
    上述函数关系具有对称性$\Delta(r_1)=\Delta(A-r_1)$,在$r_1=\dfrac{A}{2}$处求二阶导
    \[
        % \( \left.\frac{x}{y}\right|_{x=1} \)
        \left.\dfrac{\mathrm{d}^2\Delta}{\mathrm{d}r_1^2}\right|_{r_1=\frac{A}{2}}=\frac{n \left(n^2-1\right) \sin r_1}{\left(1-n^2 \sin ^2 r_1\right)^{3/2}}+\frac{n \left(n^2-1\right) \sin (A-r_1)}{\left(1-n^2 \sin ^2 (A-r_1)\right)^{3/2}}=\frac{2n \left(n^2-1\right) \sin \dfrac{A}{2}}{\left(1-n^2 \sin ^2 \dfrac{A}{2}\right)^{3/2}}>0
    \]
    从而对应极小值,代表最小偏向角。
\end{proof}

\begin{myprop}{ 掠入射}{}
	光从折射率为$n_1$的介质均匀地射入折射率为$n_2$的介质时($n_1<n_2$),出射临界角
    \[
        \theta_{c}=\arcsin\left(\dfrac{n_1}{n_2}\right)
    \]
\end{myprop}
由折射定律以及$\sin i_1\leq 1$易证。
\section{光学仪器}

\begin{myprop}{ 显微镜}{}
	设显微镜的光学筒长$\Delta=d-f_{O}-f_{E}$,人眼明视距离为$s_{0}(\approx \SI{25}{\cm})$,物镜和目镜的焦距为$f_{O}$和$f_{E}$,其角放大率为
    \[
        M=\dfrac{\Delta s_0}{f_{O}f_{E}}
    \]
\end{myprop}
证明中约定物体成像在明视距离,并使用凸透镜的牛顿公式进行化简。


\begin{myprop}{ 望远镜}{}
    设望远镜的物镜和目镜的焦距为$f_{O}$和$f_{E}
    $,则角放大率为
    \[
        M=\dfrac{f_{O}}{f_{E}}    
    \]
\end{myprop}
\begin{proof}
    使用凸透镜焦平面的特性,当平行光入射至凸透镜时(可以倾斜),其汇聚于焦平面上的一点(即\textbf{焦平面上的点共轭于无穷远}),设汇聚点距离光轴为$d$,则视场角度和原始角度之比为放大率$M=\dfrac{d/f_{E}}{d/f_{E}}=\dfrac{f_{O}}{f_{E}}$。   
\end{proof}

% \footnote{\url{https://github.com/sjtug/SJTUThesis}}。
